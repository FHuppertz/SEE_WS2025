

% Source: Section 4.2.2 Deliverable 2 

To ensure the quality of the data used for analysis, a pose filtering procedure was applied to the raw OptiTrack CSV exports.

\paragraph{Filtering Procedure:}
% TODO: If you used Chebyshev as suggested in previous assignments, keep the text below. If you used Z-Score or manual bounds, edit accordingly.
We utilized a statistical filtering approach to identify and remove outliers caused by temporary marker occlusion or tracking glitches. Specifically, we applied the Chebyshev Theorem (with a threshold of $k=20$) to the position data[cite: 107]. This method allows us to retain the majority of the data while excluding extreme deviations that do not represent physical motion.

\paragraph{Filtering Observations:}
During the filtering process, we observed the following regarding data quality:
\begin{itemize}
    \item \textbf{Outlier Frequency:} On average, we detected approximately \textbf{[INSERT NUMBER]} outliers per single experimental trial.
    \item \textbf{Total Data Loss:} The filtering process removed roughly \textbf{[INSERT PERCENTAGE]\%} of the total raw data points.
    % TODO: Add observation if specific motions (e.g., "Left") had more outliers than others due to camera angles.
\end{itemize}