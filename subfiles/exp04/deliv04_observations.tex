% 4. Discuss possible problems or error sources that can disturb the calibration process. Include any observation you may have made while testing the proper functioning of the camera with your laptop

\subsection{Observations}

\subsubsection{Camera Functionality Check}
During testing, there was no lag or focusing issues and all configurations adjustments were proper. 
Overall, the camera responded reliably for static image acquisition.

\subsubsection{Environmental Conditions}
The calibration experiment was conducted in \textit{(indoor)} conditions under \textit{(natural lighting)}. 
The illumination uniformity was verified to minimize shadow and glare on the checkerboard surface{(Captured the images by sitting opposite to the windows in the lab to minimize shadows)}. 
The camera was mounted on a laptop and images were captured without any shaking or movements. 

\subsubsection{Image Capture Process}
The board was moved to different positions and angles to provide sufficient geometric diversity for accurate calibration. 
Certain frames were \textit{(discarded)} due to \textit{(blur, board not in frame in total, poor corner detection, etc.)}. 
The process ensured that the checkerboard corners were well-detected in most images.

\subsubsection{Observed Problems and Error Sources}
Several factors that could influence calibration accuracy were identified:
\begin{itemize}
    \item \textit{(Lighting irregularities or glare)} affecting corner visibility.
    \item \textit{(Motion blur or unstable mounting device)} causing corner mismatch.
    \item \textit{(Non-planar or bent checkerboard)} introducing geometric distortion.
    \item \textit{(Incorrect checker square size input)} affecting scaling accuracy.
\end{itemize}
These observations highlight the importance of stable setup, consistent lighting, and accurate checkerboard dimensions during calibration.

\subsubsection{Reflections and Improvements}

Key takeaway: From the conducted calibration, it was observed that capturing images from diverse angles and maintaining uniform lighting significantly improved corner detection accuracy and reduced overall reprojection error. 
Future iterations could improve accuracy by \textit{(using more images, enhancing lighting uniformity, ensuring higher contrast patterns)}. 
Overall, the process provided valuable insight into the factors influencing intrinsic and extrinsic parameter estimation reliability.
