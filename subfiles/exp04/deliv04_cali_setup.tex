% 1. A description of the setup for calibration, including possible pitfalls
% 2. An estimation of the number of images and image positions required
The setup for calibration involved mounting and securing the camera to the tripod provided in the lab. Once the tripod and camera were placed in a an area with significant illumination, the next step was to work on the settings of the camera.
After disabling auto-focus and setting the camera resolution to maximum, as instructed for this experiment, the preparations were complete and the setup for the camera calibration was ready for experimentation.
Possible pitfalls include environmental factors such as changes in luminance of the checkerboard, and human errors such as inconsistent and asymmetric changes in poses for the images taken, which could introduce different focal length and other parameter calculations.
One of the most prominent example of these human errors would be changes in distance away from the camera. 
The total number of pictures taken for this experiment was 18. Considering the fact that changes in poses were applied after every other picture taken, this number was deemed to be enough.
