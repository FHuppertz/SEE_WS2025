Let $\mathbf{F}$ be a vector function, defining the \textbf{end pose} of the robot, calculated from two measured positions on a 2D plane, $\mathbf{x}_1$ and $\mathbf{x}_2$. The output components are defined such that $\mathbf{F}_1$ and $\mathbf{F}_2$ represent the position coordinates and $\mathbf{F}_3$ represents the orientation angle of the end pose. The input vectors and the functional relationship are defined in Equations \ref{eq:x1x2} to \ref{eq:F}.

\begin{eqnarray}
	\mathbf{x}_1 &=& \begin{pmatrix} x_{1,1} \\ x_{1,2} \end{pmatrix} \label{eq:x1x2} \\
	\mathbf{x}_2 &=& \begin{pmatrix} x_{2,1} \\ x_{2,2} \end{pmatrix} \\
	\mathbf{F}(\mathbf{x}_1, \mathbf{x}_2) &=& \begin{pmatrix}
		\frac{1}{2}(x_{1,1} + x_{2,1}) \\
		\frac{1}{2}(x_{1,2} + x_{2,2}) \\
		\arctan\left(\frac{x_{2,1}-x_{1,1}}{x_{2,2}-x_{1,2}}\right)
	\end{pmatrix} \label{eq:F}
\end{eqnarray}

\noindent It is important to note that the angular component, $\mathbf{F}_3$, exhibits a division by zero when $x_{1,2} = x_{2,2}$. In this specific case, the orientation is $\frac{\pi}{2} \text{ rad}$ for $x_{1,1} > x_{2,1}$, or $-\frac{\pi}{2} \text{ rad}$ for $x_{1,1} < x_{2,1}$.

As the proposed measurement mechanism utilizes two marked points, inherent errors exist in the measured coordinates. These uncertainties are characterized by the $4 \times 4$ \textbf{input covariance matrix}, $\mathbf{C}_x$, presented in Equation \ref{eq:Cx}.

\begin{equation}
	\mathbf{C}_x =
	\begin{pmatrix}
		\sigma_{x_{1,1}}^2 & 0                  & 0                  & 0                  \\
		0                  & \sigma_{x_{1,2}}^2 & 0                  & 0                  \\
		0                  & 0                  & \sigma_{x_{2,1}}^2 & 0                  \\
		0                  & 0                  & 0                  & \sigma_{x_{2,2}}^2
	\end{pmatrix} \label{eq:Cx}
\end{equation}

The assumption that the uncertainty of each coordinate is \textbf{uncorrelated} simplifies $\mathbf{C}_x$ to a diagonal matrix. The estimated standard deviation $\sigma$ for each input coordinate, is provided in Table \ref{tab:uncert}.

\begin{table}[htbp]
	\centering
	\begin{tabular}{|l|c|c|} % Added vertical pipes (|) to the column definition
		\hline % Top border
		\textbf{Parameter} & \textbf{Standard Deviation ($\sigma$)} & \textbf{Variance ($\sigma^2$)} \\
		\hline % Header/Mid border
		$x_{1,1}$          & $2.0 \text{ mm}$                       & $4.0 \text{ mm}^2$             \\
		\hline
		$x_{1,2}$          & $2.0 \text{ mm}$                       & $4.0 \text{ mm}^2$             \\
		\hline
		$x_{2,1}$          & $2.0 \text{ mm}$                       & $4.0 \text{ mm}^2$             \\
		\hline
		$x_{2,2}$          & $2.0 \text{ mm}$                       & $4.0 \text{ mm}^2$             \\
		\hline % Bottom border
	\end{tabular}
	\caption{Estimated Standard Deviation for Input Coordinates} \label{tab:uncert}
\end{table}

These estimates are primarily informed by an analysis of the major and minor sources of error detailed below:

\begin{itemize}
	\item Major Sources of Error
	      \begin{itemize}
		      \item \textbf{Pen Wiggle and Holder Flex}: \\
		            The non-rigid nature of the pen holding mechanism causes the pen tips to wobble or shift relative to the robot's chassis as the system undergoes dynamic movement, leading to inconsistent and random mark placement.

		      \item \textbf{Marking Ambiguity}: \\
		            The finite diameter of the pen tip (up to approximately 3 mm) creates an intrinsic \textbf{marking error}. The visual ambiguity in accurately estimating the true center of the resulting dot is a significant source of measurement uncertainty in the experiment.

	      \end{itemize}

	\item Minor Sources of Error
	      \begin{itemize}
		      \item \textbf{Parallax Error}: \\
		            This is a systematic human error in measurement, arising when the observer's eye is not aligned perpendicularly with the measurement scale and the marked point. This leads to inconsistent readings when quantifying the marked coordinates.
	      \end{itemize}

	\item Negligible Sources of Error
	      \begin{itemize}
		      \item \textbf{Manufacturing Tolerances of Construction Components}: \\
		            Component tolerances (e.g., LEGO bricks), typically in the range of $10-20 \, \mu\text{m}$.
	      \end{itemize}

\end{itemize}
