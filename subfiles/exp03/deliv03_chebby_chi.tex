When we use the Chebishov function to remove the outliers with using the threshold as 2.0 standard deviations, we were able to remove 7 outliers in total out of the 75 manual readings that we measured. For reference we are attaching a screenshot of the results from the code we ran to find this in Figur \ref{fig:chebyshov_results}

\begin{figure}[H]
	\centering
	\includegraphics[width=\linewidth]{figures/chebyshov_results.png}
	\caption{The results of the Chebyshov funciton removing the outliers from the manual data.}
	\label{fig:chebyshov_results}
\end{figure}




To further characterize the data's dispersion, statistical boundaries were calculated for the [Theta, $X, Y$] data for each travel direction. Confidence regions were generated for each data using two distinct methods: Chebyshev's inequality and the Chi-Squared ($\chi^2$) distribution. Figures \ref{fig:chi_square_Left_Theta} to \ref{fig:chi_square_Straight} depict these data with their corresponding boundaries.

\begin{figure}[H]
	\centering
	\begin{subfigure}[b]{0.45\linewidth}
		\centering
		\includegraphics[width=\linewidth]{figures/chi_square_Left_Theta.png}
		\caption{$\chi^2$ for Left Theta}
		\label{fig:chi_square_Left_Theta}
	\end{subfigure}
	\hfill
	\begin{subfigure}[b]{0.45\linewidth}
		\centering
		\includegraphics[width=\linewidth]{figures/chi_square_Left_X.png}
		\caption{$\chi^2$ for Left X}
		\label{fig:chi_square_Left_X}
	\end{subfigure}

	\vspace{1em}

	\begin{subfigure}[b]{0.45\linewidth}
		\centering
		\includegraphics[width=\linewidth]{figures/chi_square_Left_Y.png}
		\caption{$\chi^2$ for Left Y}
		\label{fig:chi_square_Left_Y}
	\end{subfigure}
	\caption{The figures above (\subref{fig:chi_square_Left_Theta} to \subref{fig:chi_square_Left_Y}) represent the Gaussian graph along the data of the motion of the robot going left with respective to Theta, X and Y.}
	\label{fig:chi_square_Left}
	\hfill
\end{figure}


\begin{figure}[H]
	\centering
	\begin{subfigure}[b]{0.45\linewidth}
		\centering
		\includegraphics[width=\linewidth]{figures/chi_square_Right_Theta.png}
		\caption{$\chi^2$ for Right Theta}
		\label{fig:chi_square_Right_Theta}
	\end{subfigure}
	\hfill
	\begin{subfigure}[b]{0.45\linewidth}
		\centering
		\includegraphics[width=\linewidth]{figures/chi_square_Right_X.png}
		\caption{$\chi^2$ for Right X}
		\label{fig:chi_square_Right_X}
	\end{subfigure}

	\vspace{1em}

	\begin{subfigure}[b]{0.45\linewidth}
		\centering
		\includegraphics[width=\linewidth]{figures/chi_square_Right_Y.png}
		\caption{$\chi^2$ for Right Y}
		\label{fig:chi_square_Right_Y}
	\end{subfigure}
	\caption{The figures above (\subref{fig:chi_square_Right_Theta} to \subref{fig:chi_square_Right_Y}) represent the Gaussian graph along the data of the motion of the robot going right with respective to Theta, X and Y.}
	\label{fig:chi_square_Right}
	\hfill
\end{figure}



\begin{figure}[H]
	\centering
	\begin{subfigure}[b]{0.45\linewidth}
		\centering
		\includegraphics[width=\linewidth]{figures/chi_square_Straight_Theta.png}
		\caption{$\chi^2$ for Straight Theta}
		\label{fig:chi_square_Straight_Theta}
	\end{subfigure}
	\hfill
	\begin{subfigure}[b]{0.45\linewidth}
		\centering
		\includegraphics[width=\linewidth]{figures/chi_square_Straight_X.png}
		\caption{$\chi^2$ for Straight X}
		\label{fig:chi_square_Straight_X}
	\end{subfigure}

	\vspace{1em}

	\begin{subfigure}[b]{0.45\linewidth}
		\centering
		\includegraphics[width=\linewidth]{figures/chi_square_Straight_Y.png}
		\caption{$\chi^2$ for Straight Y}
		\label{fig:chi_square_Straight_Y}
	\end{subfigure}
	\caption{The figures above (\subref{fig:chi_square_Straight_Theta} to \subref{fig:chi_square_Straight_Y}) represent the Gaussian graph along the data of the motion of the robot going right with respective to Theta, X and Y.}
	\label{fig:chi_square_Straight}
	\hfill
\end{figure}
