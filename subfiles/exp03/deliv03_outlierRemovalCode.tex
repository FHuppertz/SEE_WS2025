



This function removes outliers based on \textbf{Chebyshev’s Theorem}.
For a random variable \( X \) with mean \( \mu \) and standard deviation \( \sigma \),
Chebyshev’s inequality states that the proportion of data points lying within \( k \) standard deviations
from the mean is at least:

\[
P(|X - \mu| < k\sigma) \geq 1 - \frac{1}{k^2}
\]

In this case, we use \( k = 2 \), meaning at least \( 75\% \) of the data lie within:

\[
[\mu - 2\sigma, \, \mu + 2\sigma]
\]

Any data point \( x_i \) that satisfies \( |x_i - \mu| > 2\sigma \) is considered an outlier
and removed from the dataset.

\subsubsection*{Python Implementation}

\begin{verbatim}
def chebyshev_outlier_removal(df, column_names, threshold_sigma=2.0):
    """
    Identifies and removes outliers from the DataFrame columns based on
    Chebyshev's Theorem (all points outside mean +/- threshold_sigma * std_dev).

    Returns:
        pd.DataFrame: A new DataFrame with outliers removed.
        pd.DataFrame: A DataFrame containing the detected outliers.
    """
    df_filtered = df.copy()
    outlier_indices = set()
    print(f"\n--- Outlier Detection (Chebyshev {threshold_sigma}\u03c3) ---")

    for col in column_names:
        data = df[col].values
        mu = np.mean(data)
        sigma = np.std(data)
        lower_bound = mu - threshold_sigma * sigma
        upper_bound = mu + threshold_sigma * sigma

        # Find indices where data is outside the range
        col_outlier_indices = df[(df[col] < lower_bound) | (df[col] > upper_bound)].index
        outlier_indices.update(col_outlier_indices)
        print(f"Column '{col}': {len(col_outlier_indices)} outliers found.")

    outliers_df = df.loc[list(outlier_indices)]
    df_filtered = df.drop(list(outlier_indices))

    print(f"Total points: {len(df)}. Total outliers removed: {len(outliers_df)}. Filtered points: {len(df_filtered)}")
    return df_filtered, outliers_df
\end{verbatim}